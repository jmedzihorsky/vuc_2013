\documentclass[11pt, a4paper]{article}

\usepackage{a4wide}

\usepackage{datetime}
\usepackage{hyperref}
\usepackage{booktabs}
\usepackage{float}

\usepackage{charter}
\usepackage[T1]{fontenc}


\title{VUC 2013 Elections Dataset Codebook}
\author{
	Juraj Medzihorsky \\
	\href{mailto:medzihorsky_juraj@ceu-budapest.edu}{\texttt{medzihorsky\_juraj@ceu-budapest.edu}}
}
\date{\ddmmyyyydate\today~at~\currenttime}

\begin{document}

\maketitle

\noindent
\textbf{Please cite the data \& codebook as:}

\noindent
Medzihorsky, Juraj. (2013) \textsl{VUC 2013 Elections Dataset}. version: \ddmmyyyydate\today~at~\currenttime

\tableofcontents

\section{Source}

Official elections results were scraped from \href{http://osk2013.statistics.sk/}{http://osk2013.statistics.sk/}.


\section{Elections}

\begin{itemize}
	\item	First round date:  9 November 2013
	\item	Second round date:  23 November 2013.
\end{itemize}

The second round took place in the following five
self-governing regions:

\begin{itemize}
	\item	Banska Bystrica		
	\item	Bratislava
	\item	Kosice
	\item	Nitra
	\item	Trnava
\end{itemize}


\section{Coding}

Variable names are modular, see Table~\ref{tab:modules} for details.

\begin{table}[H]\footnotesize
	\label{tab:modules}
	\caption{Basic variable name elements.}
	\begin{center}
		\begin{tabular}{ll}
			\toprule
			Element	& Description \\		
			\midrule
			\texttt{r1\_}	&	data specific to first round \\
			\texttt{r2\_}	&	data specific to second round \\
			\texttt{\_g\_}	&	data specific to gubernatorial elections; either \texttt{votes} or \texttt{name} \\			
			\texttt{\_c\_}	&	data specific to council elections; either \texttt{votes} or \texttt{name} \\						
			\texttt{name\_}	&	a name; if followed by a number a name of a candidate, otherwise a territorial name \\
			\bottomrule
		\end{tabular}
	\end{center}
\end{table}

Table~2 shows variable name elements specific to turnout variables.
Note that there is only one envelope also when there are two ballots -- governor and council
-- in the first round.

\begin{table}[H]\footnotesize
	\label{tab:turn}
	\caption{Variable name elements specific to turnout.}
	\begin{center}
		\begin{tabular}{ll}
			\toprule
			Element	& Description \\		
			\midrule
			\texttt{registered\_voters}	&	number of voters on rolls \\
			\texttt{envelopes\_given}	&	number of envelopes picked-up by voters\\
			\texttt{envelopes\_valid}	&	\texttt{\_governor} or \texttt{\_council} \\			
			\bottomrule
		\end{tabular}
	\end{center}
\end{table}



Table~\ref{tab:trans} shows the English translations of Slovak terms used in the coding. 

\begin{table}[H]\footnotesize
	\label{tab:trans}
	\caption{Slovak \& English terms.}
	\begin{center}
		\begin{tabular}{ll}
			\toprule
			Slovak & English \\
			\midrule
			samospravny kraj &	(self-governing) region \\
			vyssi uzemno-spravny celok (VUC) & (self-governing) region \\
			predseda VUC & governor \\
			zastupitelstvo VUC & regional council \\
			clen zatupitelstva VUC & councilor \\
			kraj	&	region \\
		 	obvod	&	district \\
			obec 	&	municipality	\\
			okrsok	&	ward/polling station \\
			\bottomrule
		\end{tabular}
	\end{center}
\end{table}


Since ballot length varies widely across districts, the number of columns for candidate-related
variables -- \texttt{\_name\_} \& \texttt{\_votes\_} -- is set to the district with the longest ballot.
As a result there are plenty of \texttt{NA}. These are not missing data.








\section{Files}

There are two files:
\begin{itemize}
	\item	\href{https://github.com/jmedzihorsky/vuc_2013/blob/master/data/vuc_2013_ward.dat}{\texttt{vuc\_2013\_ward.dat}} where each row is a polling station (ward).
	\item	\href{https://github.com/jmedzihorsky/vuc_2013/blob/master/data/vuc_2013_municipality.dat}{\texttt{vuc\_2013\_municipality.dat}} where each row is a municipality.		
\end{itemize}
Both files are in tab-separated format.

\end{document}
